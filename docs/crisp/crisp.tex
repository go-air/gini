\documentclass{article}
\usepackage{amsmath}

\newcommand\proto[1]{$\langle\mbox{#1}\rangle$}

\begin{document}
\bibliographystyle{plain}
\title{CRISP: The CompRessed Incremental SAT Protocol}
\author{Scott Cotton, IRI France, SAS}
\begin{abstract}
We present CRISP, an extensible efficient wire protocol for incremental SAT solving.
The protocol allows applications to communicate with a remote or local server over
tcp or unix domain sockets.  It efficiently transfers data andminimizes unnecessary round
trips.  It is simple, and easily extensible to domains such as optimisation, quantifiers,
proofs, etc.  An open source reference implementation is available and discussion for
future versions and extensions takes place in public.

\end{abstract}
\maketitle
\section{Introduction}
SAT solvers are traditionally linked against applications, and very often the 
application context yields a high ratio of very easy problems to medium or
hard problems.  "Big gun" SAT solvers are often NOT used in applications because
of the code complexity and/or hardware requirements.  Additionally, SAT solvers
are usually written in C or C++, whereas higher level application contexts may involve
a software stack with many languages and components.  The inter-language bindings
create integration and computational overhead.  To address these issues and 
at once provide a useful piece of infrastructure for cloud and distributed deployment,
we designed CRISP, the CompRessed Incremental SAT Protocol. 

The base protocol can be seen as a wire protocl client-server version of the
standard incremental SAT solving interface originally defined in
\cite{DBLP:journals/entcs/EenS03}.

The rest of this paper is organised as follows.  Section \ref{section:protocol}
details the wire protocol.  Section \ref{section:ext} defines the generic
extension mechanism and two useful extensions (optimisation and assumption
based multi-plexing). Section \ref{section:exp} gives some experimental data.
Section \ref{section:conc} concludes.


\section{The Protocol}
\label{section:protocol}

\subsection{Wire Data}

Every piece of data communicated between the client and the server takes the
logical form of a uint32, an unsigned 32 bit value.

This space houses literals and communication instructions/directives between
the client and the server.  These instructions/directives are called
protocol points in the following.

We only use $17$ protocol points, but to enable extensions, we
reserve 256 integers for protocol points, at the high end of the range
representable by uint32. The rest of the space is used to house variables
and literals, coded in the tradition of SAT solvers (see Adding below).

\subsection{Flow Overview}
\label{section:flow}

Here we present an overview of the interactions in CRISP.

\begin{enumerate}
\item Client negotiates connection with server
\item Client then requests (\proto{add} or \proto{assume}) as many times as it likes.
 The Server does not respond to these requests.
 \item Client then requests \proto{solve}.  This enters a loop between the client and the server on the same connection as
 follows:
	 \begin{enumerate}
		 \item Client: \proto{solve}
	 \item Server: \proto{unknown}$|$\proto{sat}$|$\proto{unsat}$|$\proto{end}
	 \item Client: \proto{continue}$|$\proto{end}
	 \end{enumerate}
 \item Client sends (\proto{model} or \proto{modelfor} or \proto{failed} or \proto{failedfor}) as many times as it likes.
 Each time it sends one of these operations, the server responds with the requested data.
 \item Optionally, the client may send \proto{reset}.  In this case, the server forgets added clauses and
	 the flow goes to item $2$ above.
 \item Client sends \proto{quit}, both ends disconnect
\end{enumerate}

In the above, step 3,

\begin{center}
 (b,c) repeats until server sends \proto{sat}, \proto{unsat}, or \proto{end}
\end{center}

Thus step 3 can be represented by the regular expresion below, where vertical alignment 
represents concatenation.
\begin{center}
\begin{tabular}{|l|}
\hline \\
client:\proto{solve} \\
(server:\proto{unknown} (client:\proto{continue}$|$client:\proto{end})$^\star$ \\
(server:\proto{end}$|$\proto{sat}$|$\proto{unsat})\\
\hline
\end{tabular}
\end{center}


The meaning of the server sending \proto{unknown} is that it doesn't know the
answer and is willing and able to make more progress solving the problem.
The maning of the server initiating \proto{end} (in response to client \proto{solve}
or client \proto{continue}) that it unable or unwilling to
make more progress on solving the problem and does not know the answer. 

In this loop, the client does not need to wait between reads and writes,
since presumably the server is doing the solving.  On the other hand, the
server needs not try to respond so fast as to eat up unnecessary resources
So the server may do this at an appropriate frequency.  

If we call the regular expression above S, the allowable overall (error-free)
flow interactions can be represented by the regular expression

\[1  ((2^{\star}  S  4^{\star})^{\star} 5?)^{\star} 6\]

There are also error conditions, corresponding to further constraints on the
allowable flows;  these simply abort the sequence and cause a disconnect; these
are not represented in this overview but rather are detailed below.

\subsection{Varuint Encoding}
The wire format then uses var-uint encoding (a stream of bytes each of which
indicates whether it is the last with 1 bit and uses 7bits to encode the
non-zero LSBs of the value).  This is a standard variable width encoding which
compresses values which tend to be small.  This works as follows.

Let $u_i$  for $i$ in $0..31$ be the bits of an unsigned 32 bit int u.  The encoding
of $u$ is a slice of bytes $e= e_0 e_1 \ldots $.

\begin{center}
\begin{tabular}{l}
 $e_0\&127$ houses the $7$ most LSB of $u$: here denoted $u_{0..7}$\\
~~~~If $u \gg 7 = 0$, then the length of $e$ is $1$.\\
Otherwise $e_1\&127$ houses $u_{7..14}$.\\
~~~~If $u \gg 14 = 0$, then the length of $e$ is $2$.\\
Otherwise $e_2\&127$ houses $u_{14..21}$.\\
...\\
\end{tabular}
\end{center}

This continues for all $32$ bits.  Also, For every element $e_i$ of e except the last, $e_i\&128 \neq 0$.

The server may decode a stream of varuint32 data and then, for
each value $v$, test whether it is a command/code point by testing

\[v >= (2^{32}-1) - 255\]

if true, $v$ is a command, if false, $v$ is a literal.

\subsubsection{Rational}
We distinguish the special value '0' as a literal to
zero-terminate clauses and lists of assumptions or lists of failed literals.
This is conventional and only requires 1 byte in varuint format.  Commands
happen with much less frequency in the protocol when there is a potential
need to send lots of data fast (eg loading a big dimacs).  Commands are
large values and hence have larger varuint size.

\subsection{Protocol Points}

We reserve 256 code points for extensions to learning, optimisation, etc.
In this proposal we have only the following op codes


\begin{center}
	\begin{tabular}{|l|l|}
		\hline
		\proto{add} & $(2^{32}-1)$ \\
		\proto{assume} & $(2^{32}-1) - 1$\\
		\proto{solve} & $(2^{32}-1) - 2$\\
		\proto{continue} & $(2^{32}-1) - 3$\\
		\proto{end} & $(2^{32}-1) - 4$ \\
		\proto{error} & $(2^{32}-1) - 5$ \\
		\proto{failed} & $(2^{32}-1) - 6$\\
		\proto{failedfor} & $(2^{32}-1) - 7$\\
		\proto{model} & $(2^{32}-1) - 8$ \\
		\proto{modelfor} & $(2^{32}-1) - 9$ \\
		\proto{sat} & $(2^{32}-1) - 10$ \\
		\proto{unsat} & $(2^{32}-1) - 11$ \\
		\proto{unknown} & $(2^{32}-1) - 12$ \\
		\proto{quit} & $(2^{32}-1) - 13$\\
		\proto{reset} & $(2^{32}-1) - 14$\\
		\proto{key} & $(2^{32}-1) - 15$\\
		\proto{ext} & $(2^{32}-1) - 16$\\
		\hline
\end{tabular}
\end{center}

The remaining protocol points are for specific extensions.

\subsection{Variables and Literals}

Variables are Boolean (can be either true/false) and each variable is
associated with a uint32.  A "literal" in propositional logic is just a
variable or the logical negation of a variable.  CRISP uses standard
SAT-solver encoding of variables and literals.

If a variable is indicated by some uint32, say $x$, then a literal $m$ over
$x$ is

$$\begin{array}{ll}
	x \ll 1 &                \mbox{if }m\mbox{ is positive}\\
	(x \ll 1) | 1 &            \mbox{if }m\mbox{ is negative}\\
\end{array}$$

Since we have uint32 coding, and 256 coding points, the maximum variable
representable in the protocol is

 $$((2^{32}-1) - 256) \gg 1$$

or equivalently

 $$2147483519$$

 \subsection{Connection negotiation}

Upon connecting, the server replies with 
\begin{enumerate}
	\item "CRISP" (as 5 uint32s coding the ASCII values of 'C', 'R', 'I', 'S', 'P');
	\item followed by a uint32 $v$ indicating the protocol version number which is comprised of a
major version and a minor version.  The major version is the upper 8 MSB
of $v$ and the minor protocol version is the 24 LSB of $v$.
\end{enumerate}

Some servers may want to protect access by requring a key.  If they do this,
then they can simply wait for the first op from the client.  If the first op
is not \proto{key}, then the server can disconnect the client.

Clients connecting to \proto{key}-passed servers can send the \proto{key} op
followed by the key length in terms of uint32 atoms (the key must be 4-byte
aligned).  Then the client sends the key.  After receiving the key, the server
could just disconnect if it doesn't accept the client.

\subsection{Adding}

Adding adds permanent constraints to the solver on the server side.
Each constraint is in the form of a clause, which is a disjunction of
literals.

When the client adds clauses, it sends the \proto{add} op followed by
a list of clauses, where each clause is a null-terminated list
of literals.  When it is done adding clauses, it sends a \proto{end}.

The server does not respond to \proto{add} or \proto{end} ops.

\subsection{Assuming}

When the client makes assumptions, it sends the \proto{assume} op followed by
a null terminated list of literals.  The server does not respond to this
op.

The server takes into account the assumptions temporarily, only for the next
call to solve.  Subsequent calls to solve after the next call are not
effected.

\subsection{Solving}

Solving is the only part of the protocol which involves several
round trips between the client and the server.

After the client sends \proto{solve}, it reads from the server in a blocking read
until the server sends a response.  The response is either \proto{sat}, \proto{unsat},
\proto{unknown}, or \proto{end}.

 If the response is \proto{unknown}, then the client must respond with either
 \proto{continue} or \proto{end}.  If the client sends \proto{continue},  the protocol enters
 the same state as if the client had just sent \proto{solve}.  If the client sends
 \proto{end}, the protocol exits the \proto{solve} state; the client can add more clauses
 or make more assumptions and try to solve again (or quit).

 If the server response is \proto{sat},\proto{unsat}, or \proto{end} then client must not respond
 with \proto{continue} or \proto{end}.  Instead, the client may optionally request models
 if the response was \proto{sat}, or optionally request failed assumptions if the
 response was \proto{unsat}.  In any event, the client may continue to \proto{add} or
 \proto{assume} or may \proto{quit}.

\subsection{Models}

Models are valuations of variables in a problem which satisfy the permanent
constraints and assumptions in the previous \proto{solve} interaction.  CRISP
supports 2 mechanisms for models: partial and complete models, corresponding
respectively to the \proto{modelfor} and \proto{model} operators.

The \proto{model} and \proto{modelfor} operators can only be sent if the previous \proto{solve} operator
ended with a \proto{sat} operator from the server and no \proto{add} or \proto{assume} has taken place since.
Otherwise, the server sends \proto{error} and disconnectes.

Complete models are obtained using the \proto{model} operator.  The client
sends \proto{model} and the server responds with a truth value for every variable,
in the order of variable index/id 1,2,3,.... up until the maximum
variable used in any added clause or assumption.

The encoding of the truth values is the same for partial and complete models,
and is described below.

Partial models are obtained by using the \proto{modelfor} operator.  The
client sends \proto{modelfor} followed by a null terminated list of literals.
The server responds with the truth value for each of these literals in
the same order specified specified by the client.

In both \proto{model} and \proto{modelfor}, the list of truth values is encoded by the
server as follows.  First, the server sends a uint32 indicating how many
uint32s are to be sent subsequently to communicate the model.  Then:

Let $N$ be the length of the list, and let let $M$ be $\lceil{N/32}\rceil$.
 Then $M/32$ uint32 values are sent by the server.  Let $u_i$ be the $i$'th such value.
 The truth value for element $j$ of the list is

 $$u_{j/32} \& (1 \ll j\bmod 32) \neq 0$$

Thus models are communicated in a standard compressed bitvector representation
with $32$ bit word size.

\subsection{Failed Assumptions}
Failed assumptions are a subset of the assumptions from the last \proto{solve}
sequence which are sufficient to render the problem unsat.  Like the \proto{model}
and \proto{modelfor} operators, there are 2 failed operators, \proto{failed} and
\proto{failedfor}.

\proto{failed} retrieves a sufficient subset of all previously
assumed literals to render the last problem posed to \proto{solve} unsat.

\proto{failedfor} specifies a list of assumptions which are of interest to the
client.  The server responds with the maximal subset of this list which
intersects what would be returned by \proto{failed}.

Thus, when the client sends \proto{failedfor} it follows this operator with
a null terminated list of literals. But when the client sends \proto{failed}
it immediately waits the response from the server.

In both cases, the server replies with a null terminated list of literals.
If the server replies to \proto{failedfor}, the list of literals in the response
from the server must respect the order of the list provided by the client
in \proto{failedfor}.

\subsection{Error Conditions}

Errors are always sent from the server to the client.  Each \proto{error} op
is followed by a single uint32 value, which encodes more information
about the error.

Anytime the server responds with an \proto{error}, the server subsequently
simply closes the connection.

The following are common errors:
\begin{enumerate}
	\item If the varuint encoding encodes a number outside the uint32 range
then the server responds with an error.  The error code is 1.

\item If the client has this error reading from the server, then it must disconnect.

\item If a client requests a model or partial model and the previous solve dis
not end with \proto{sat}, the server responds with an error.  The error code is 2.

\item If a client requests failed assumptions and the previous solve was not unsat, the
server responds with an error, the error code is 3.

\item If a client requests failed assumptions or model and an \proto{add} or an \proto{assume}
has taken place since the last \proto{solve}, the server responds with an error
whose code is 4.

\item If a client sends an unknown opcode, the server responds with an error
	whose code is 5.  

\item If the client reads an unknown opcode, it must immediately disconnect.

\item If a protocol opcode is read when data is expected, we have error code 6.

\item Other generic server errors are called "internal server errors", and have
error code 7.
\end{enumerate}

\subsection{Reseting}
The \proto{reset} opcode causes the server to forget all added clauses and assumptions,
{\em i.e} to go to the state just after connection negotiation.  This is useful
in a wire protocol because the overhead of creating a connection is higher than
the overhead of creating/initializing an incremental SAT solver via API.

\section{Extensions}
\label{section:ext}
In this section, we mention several possible extensions to CRISP.  We have 
not yet implemented these, but rather present them to show how easy it 
is to adapt CRISP to different Boolean reasoning and optimisation tasks 
and solving requirements.

\subsection{Extension negotiation}
The \proto{ext} opcode indicates a request from the client to find
available extensions on the server.  Each extension has 
\begin{enumerate}
	\item A unique 16 bit identifier.
	\item A list of opcodes.
\end{enumerate}

Thus, in response to \proto{ext}, the server responds with a null terminated list 
\[
	E = [(i_0, n_0), (i_1, n_1), \ldots]
\]
where each $(i,n)$ is placed into a uint32 ($i$ is the 16 MSB, $n$ is the 16 LSB).  
Value $(i_j, n_j)$ indicates that extension with identifier $i_j$ has $n_j$ opcodes.

To map extension opcodes to server instances, the client then computes that opcode
$c$ (zero indexed) in extension at position $i$ in $E$ is
\[
	(2^{32}-1) - 255 + \Sigma_{j<i} n_j + c
\]

This allows different servers to have different sets of extensions all packed into
the $240$ extension opcode slots, and it allows clients to map such opcodes
\footnote{This mechanism does require that extensions have unique identifiers.  We recommend
filing an issue with the Gini project on github to request extension identifiers 
and definitions.  Short of that, one could always just choose a random number for the
a new extension identifier to minimize the chances of collision.}.

\subsection{Optimisation}

There are many forms of Boolean optimisation, such as MaxSAT where the problem
is to find the maximal number of satisfiable clauses, possibly weighted.  Other
forms include pseuodo-Boolean optimisation, in which constraints and objectives
can express the number of literals in clauses which are true.  All of these
forms can be translated to each other, see for example
\cite{DBLP:journals/jar/LiffitonS08} or \cite{DBLP:journals/jsat/EenS06}.

In this extension, we present a tiny, minimalistic incremental Boolean
optimisation interface which can also be used to solve a wide variety of
Boolean optimisation forms provided the problems are coded accordingly.

This extension defines $5$ new opcodes, in the following order.

\begin{enumerate}
	\item \proto{addmax}.  This opcode adds a literal to the set of literals for which
		the solver searches for a maximal satisfying assignment.  It is used by 
		the client anywhere where \proto{add} or \proto{assume} can be used.
		There is no response from the server.
	\item \proto{clearmax} This opcode clears the set of literals to maximize.
		It is used by the client anywhere \proto{add} or \proto{assume} can be used.
		There is no response from the server.
	\item \proto{nextlower} This opcode directs the solver to find the next 
		lower bound on the number of maximisation literals which can be satisfied.
		It has semantics like \proto{solve} in that it enters a communication
		loop between the client and the server.  That loop behaves exactly as
		in \proto{solve}, except it replaces \proto{sat} with \proto{better} (see below).
	\item \proto{nextupper} This opcode directs the server to find the next
		upper bound on the number of maximisation literals which can be satisfied.
		Otherwise, it operates exactly like \proto{nextlower}.
	\item \proto{better}.  This opcode replaces \proto{sat} in a solver communication
		sequence for \proto{lowerbound} or \proto{upperbound}. It is sent from 
		the server when a satisfying assignment increases the lower or upper bound
		in response to \proto{nextlower} or \proto{nextupper}. Unlike \proto{sat}, 
		It is followed by the respective bound, coded as another uint32.  Also
		unlike \proto{sat}, it does not undo the assumptions.
\end{enumerate}

\subsection{Assumption Multiplexing}
One interesting possibility which arises when interacting with a SAT problem is
partitioning and multiplexing single hard problem instances accross several
cores or machines.  In this section we present an extension which allows
partitioning problems with assumptions and asking the server to multiplex the
solving accross different sets of assumptions.  

\subsubsection{Opcode \proto{mux}}
This extension introduces a new opcode called \proto{mux}, which is used by the
client and the server to multiplex exchanges on top of the base protocol.  In
principal, it can also be used for other extensions, though we do not detail
that here.

The \proto{mux} opcode is a sort of meta code in that it is (almost) always followed by
an identifier $i$ and another opcode $o$.  The identifier is used to identify which of several
multiplexed operations are involved in subsequent exchanges between the client
and the server.  The client uses \proto{mux} only for assumptions and solving.
For example, to request multiplexed solving under $x$ and $\neg x$, the
client might send 

\begin{tabular}{l}
	\proto{mux} $1$ \proto{assume} $x$ $0$\\
	\proto{mux} $2$ \proto{assume} $\neg x$ $0$\\
	\proto{mux} $1$ \proto{solve}\\
	\ldots \\
	\proto{mux} $2$ \proto{solve}\\
\end{tabular}

After a send of \proto{mux} $n$ \proto{solve}, the client interacts with the
server for the solve sequence as usual ({\em i.e.} without \proto{mux}), with the following
exceptions.  

\begin{enumerate}
	\item The server prefixes all immediate post-solve replies with \proto{mux} $n$
		for the appropriate $n$, this includes \proto{sat}, \proto{unsat}, \proto{unknown} and 
		\proto{end}\footnote{In fact, it is not strictly necessary for the server to provide this
		information because the response sequence is a deterministic function of the client request
	sequence, but providing this information simplifies client implementations and provides 
  the possibility of an additional sanity check}.

  \item The client must respond to \proto{mux} $n$ \proto{sat} with \proto{model}, \proto{modelfor}, or
		\proto{end}.  If the client responds with \proto{end}, the server must not reply.

	\item The client must respond to \proto{mux} $n$ \proto{unsat} with \proto{failed} \proto{failedfor} or
		\proto{end}.  If the client responds with \proto{end}, the server must not reply.

	\item The client may, in addition to sending \proto{continue} or \proto{end}
		as usual in response to a server supplied \proto{mux} $n$ \proto{unknown}, either
		\begin{enumerate}
			\item send a new \proto{mux} $m$ \proto{solve}
		for some $m$ which is not currently in a solve interaction but for which there
		are assumptions defined.  This tells the server to continue solve sequence $n$ and also 
		multiplex a new solve sequence $m \neq n$. 
	\item or, the client may send \proto{mux} $m$ \proto{assume} for some $m$ not in
		a solve sequence.  This tells the server to continue solve sequence $n$ and also
		to record a new set of assumptions.
\end{enumerate}
\end{enumerate}

In this way, the client may queue parallel solving piecewise to the server, by making
a request for $2$ solve steps in response to a \proto{unknown}.

\subsubsection{Example Trace}
To continue the example above, the following is a possible trace between the
client and the server exercising the various options above.

\begin{tabular}{|llll|}
	\hline
	N & {\bf who} & {\bf message} & {\bf queue (mux-id, step)}\\
	\hline
	1 & client & \proto{mux} $1$ \proto{assume} $x$ $0$ & $\emptyset$\\
	2 & client & \proto{mux} $2$ \proto{assume} $\neg x$ $0$ & $\emptyset$ \\
	3 & client & \proto{mux} $1$ \proto{solve} & $[(1,1)]$ \\
	4 & server & \proto{mux} $1$ \proto{unknown} & $\emptyset$ \\
	5 & client & \proto{continue} & $[(1,2)]$ \\
	6 & server & \proto{mux} $1$ \proto{unknown} & $\emptyset$ \\
	7 & client & \proto{mux} $2$ \proto{solve} & $[(2,1), (1,3)]$ \\
	8 & server & \proto{mux} $2$ \proto{unknown} & $[(1,3)]$ \\
	9 & client & \proto{mux} $3$ \proto{assume} $y$ $0$ & $[(1,3), (2,2)]$\\
	10 & server & \proto{mux} $1$ \proto{unknown} & $[(2,2)]$\\
	11 & client & \proto{mux} $3$ \proto{solve} & $[(2,2),(3,1),(1,4)]$\\
	12 & server & \proto{mux} $2$ \proto{sat} & $[(3,1),(1,4)]$\\
	13 & client & \proto{model}               & $[(3,1), (1,4)]$ \\
	14 & server & \ldots $\neg x$ \ldots  & $[(3,1), (1,4)]$\\
	15 & server & \proto{mux} $3$ \proto{unsat} & $[(1,4)]$ \\
	16 & client & \proto{end} & $[(1,4)]$\\
	17 & server & \proto{mux} $1$ \proto{unknown} & $\emptyset$\\
	18 & client & \proto{end} & $\emptyset$\\
	19 & server & \proto{mux} $1$ \proto{end} & $\emptyset$\\
	\hline
\end{tabular}

\subsubsection{Work Queue -- Order and Size}
With muxing, the server maintains a fine grained work queue which is ultimately
completely controlled by the client.  There is exactly one interaction in the protocol which 
allows the client to request more than one response from the server, which is when the client
responds to \proto{mux} $m$ \proto{unknown} with \proto{mux} $n$ \proto{solve}.  
Every time this happens, the server appends the next round of work for $m$ to the queue
followed by the first round of work for $n$.  The queue is a FIFO and whenever the server 
has an outstanding work queue and is not expecting a response from the client, the server pops
the head of the queue and sends the result.  

To keep track of when to expect responses from the server, the client simply
needs to keep a counter of the size of the work queue, which increments by $1$
on the initial send of \proto{mux} $n$ \proto{solve}, increments by $2$ every
time the client sends this in response to \proto{mux} $n$ \proto{unknown}, and
decrements by $1$ every time the client receives a muxed response from the
server.  For example, after line $14$ in the trace above, the client knows
that there are 2 elements in the work queue, so it waits to read from the server
before sending any new requests.  Similarly, the server knows that whenever the
work queue is non-empty and it sent a response for a model or failed literals,
the client will not make a request.

\subsubsection{Mux Queue bounds}
By convention, the client may send \proto{mux} $0$ to the server to query
the maximal size of the work queue.  The server responds with a single uint32.
This number may be useful for partitioning algorithms.

\subsubsection{Mux Summary}
In summary, in terms of the overall flow of the base protocol outline in
Section \ref{section:flow}, muxing preserves a similar flow and corresponding
constraints for each multiplexed solution (and extraction) interaction.
However, the client {\em must} respond to \proto{sat} and \proto{unsat} even it
is does not expect a model.  This is necessary so that after each step of the
protocol who reads and who writes to the connection is well defined and there
is only one reader.

\subsection{Clause Sharing}
A common approach to parallel sat solving is running independent solvers with
different configurations or orderings of data structures and then to have them
share learned clauses.  It would be beneficial to have an extension which
enabled this to work in a distributed fashion as well.  In fact, this has
already been implemented and evaluated with mpi (message passing interface).
However, mpi is meant for scientific cluster computing and each implementation
uses different message encodings.  A simple agreed upon protocol for clause
sharing would enable different solvers to interact in a wider variety of
computing environments. 

Hence, we propose a crisp extension called clause sharing.  The extension
uses the \proto{add} encoding of clauses to publish and retrieve clauses.
We define 4 new protocol points for this extension.
\begin{enumerate}
	\item \proto{share} This protocol point is injected by the client
		immediately after she is done sending \proto{add}.  It is a directive
		to the server to share clauses with the client for the problem defined
		by the previous \proto{add}.  The server responds with \proto{ok} if
		it is willing to run a solver or communicate with another solver to share
		clauses.  Otherwise, the server responds with \proto{end}.
	\item \proto{putc} This protocol point is injected by the client
		during a solve loop, whenever the client may otherwise send
		\proto{continue}.  It is followed by a set of clauses encoded
		as in \proto{add} (ie ended with \proto{end}).  The server simply interprets this as an
		implicit \proto{continue} and may answer as in the core CRISP
		protocol

	\item \proto{getc}
		This protocol point is also injected by the client whenever she
		may otherwise send \proto{continue}.  The server responds with
		a set of clauses encoded as in add, ending with \proto{end}. The
		server may also respond with \proto{sat}, \proto{unsat}, or \proto{end}
		instead of with a set of clauses.
\end{enumerate}

In this case, the client may in fact be a solver.  The server may also be 


\section{Experiments}
\label{section:exp}
We measured the sizes of varint encoding based compression compared to
dimacs format and gzip'd dimacs format.


\section{Conclusion}
\label{section:conc}
We have presented an efficient wire protocol for incremental sat solving which 
can be used to relegate SAT solving in applications to an external server,
possibly with dedicated hardware.  The protocol is simple, compact, minimizes
round trips, provides compression, and is easily extensible to various domains.
From the client point of view, it can be used as a drop in replacement for incremental 
sat applications.  

From a solver developer point of view, the server side is completely isolated
from the calling application.  As a result, a solver developer can use 
arbitrary hardware, programming language or style, and make arbitrary use of 
operating system resources independent of the application.

Moreover, the protocol provides a simple, useful piece of infrastructure for
distributed solving, as any solvers which are integrated into a server (or can
serve the protocol themselves) can be used to solve sub problems of a hard
problem with assumptions. Moreover, the client-server paradigm makes this
composable: the server can again be a client to make use of other servers.

We would like to thank Daniel Leberre and Armin Biere for helpful discussion
during the development of CRISP.

\bibliography{crisp}
\end{document}
